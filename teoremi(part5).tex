% !TEX root = C:\Users\Agostino\Analisi\teoremi(part5).tex
\documentclass{article}
\usepackage{graphicx} % Required for inserting images
\usepackage{amsmath}
\usepackage{amsfonts}
\usepackage{tcolorbox}

\title{Teoremi}
\author{Agostino Cesarano}
\date{January 2024}

\begin{document}

\maketitle
\newtcolorbox{calloutbox}{
    colback=red!5!white,
    colframe=red!75!black,
    fonttitle=\bfseries,
    title=Attenzione
}

\setcounter{part}{4}
\part{De L'Hôpital e Taylor}
\section{Teorema di De L'Hôpital}
Siano f e g due funzioni derivabili in un intorno del punto $x_{0}$ tale che
% lim di f(x) e g(x) per x che tende a x_{0} sono entrambi infiniti o entrambi 0
\begin{equation*}
    \lim_{x \to x_{0}} f(x) = \lim_{x \to x_{0}} g(x) = 0 \quad \text{o} \quad \lim_{x \to x_{0}} f(x) = \lim_{x \to x_{0}} g(x) = \pm \infty
\end{equation*}
se $g(x), g'(x)\neq 0$ in un intorno di $x_{0}$, allora
\begin{equation*}
    \lim_{x \to x_{0}} \frac{f(x)}{g(x)}= \lim_{x \to x_{0}} \frac{f'(x)}{g'(x)}
\end{equation*}
$f, g$ continue in $x_{0}$

$$
    \lim\limits_{ x \to x_{0} } \frac{f'(x)}{g'(x)} = \frac{f'(x_{0})}{g'(x_{0})}
$$
per ipotesi abbiamo che $f(x_{0})=g(x_{0})= 0$
$$
    \lim\limits_{ x \to x_{0} } \frac{f(x)}{g(x)} = \lim\limits_{ x \to \infty } \frac{f(x)-f(x_{0})}{g(x)-g(x_{0})} = \lim\limits_{ x \to x_{0} } \frac{\frac{f(x)-f(x_{0})}{x-x_{0}} }{\frac{g(x)-g(x_{0})}{x-x_{0}}} = \frac{f'(x)}{g'(x)}
$$
dividendo numeratore e denominatore per $x-x_{0}$ è da li otteniamo il \textbf{rapporto incrementale}, quindi la derivata di $f(x)$ e $g(x)$.
\newpage
\section*{Sviluppo di Taylor}
Sia $f(x)$ una funzione derivabile $n$ volte in un intorno di $x_{0}$, allora la funzione $f(x)$ può essere sviluppata in serie di Taylor centrata in $x_{0}$ come segue:
\begin{equation*}
    f(x) = f(x_{0}) + f'(x_{0})(x-x_{0}) + \frac{f''(x_{0})}{2!}(x-x_{0})^{2} + \frac{f'''(x_{0})}{3!}(x-x_{0})^{3} + \ldots + \frac{f^{(n)}(x_{0})}{n!}(x-x_{0})^{n} + R_{n}(x)
\end{equation*}
\section{Teorema di Taylor con resto di Peano}

Il resto di Peano centrato in $x_{0}$ è anche detto \textbf{Resto di Mc Laurin}

Sia una funzione $f$ derivabile $n-volte$ in $x_{0}$ allora $$ f(x) =
    \sum_{k=0}^n \frac{f^{(k)}(x_{0})}{k!}(x-x_{0}) + R_{n}(x) $$ con $$
    \lim\limits_{ x \to x_{0} } \frac{R_{n}(x)}{(x-x_{0})^n} = 0 $$
Sappiamo che $$R_{n}(x) = \lim\limits_{ x \to x_{0} }
    \frac{R_{n}(x)}{(x-x_{0})^n} $$
sostituiamo $R_{n}(x)$ nella formula del limite
$$ \lim\limits_{ x \to x_{0} } \frac{f(x) - \sum_{k=0}^n
    \frac{f^{(k)}(x_{0})}{k!}(x-x_{0})^{k}}{(x-x_{0})^n}$$
    per ipotesi sappiamo che deve essere uguale a 0, deriviamo numeratore e denominatore: $$
    \lim\limits_{ x \to x_{0} } \frac{f'(x) - \sum_{k=1}^n
    \frac{f'^{(k)}(x_{0})}{k!}(x-x_{0})^{k-1}}{n(x-x_{0})^{n-1}} $$ $$
    \lim\limits_{ x \to x_{0} } \frac{f''(x) - \sum_{k=2}^n
    \frac{f''^{(k)}(x_{0})}{k!}(x-x_{0})^{k-2}}{n(n-1)(x-x_{0})^{n-2}} $$

questo si può ripetere $n-1$ volte $$ \lim\limits_{ x \to x_{0} }
    \frac{f^{n-1}(x) - f^{(n-1)}(x_{0}) - f^{(n)}(x_{0})(x-x_{0})}{n!(x-x_{0})} =$$
$$\frac{1}{n!} \lim\limits_{ x \to x_{0} } \left[ \frac{f^{n-1}(x) -
    f^{(n-1)}(x_{0}) - f^{(n)}(x_{0})(x-x_{0})}{x-x_{0}} \right] = 0 $$
    

\end {document}
