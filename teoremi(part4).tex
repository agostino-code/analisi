% !TEX root = C:\Users\Agostino\Analisi\teoremi(part4).tex
\documentclass{article}
\usepackage{graphicx} % Required for inserting images
\usepackage{amsmath}
\usepackage{amsfonts}
\usepackage{tcolorbox}

\title{Teoremi}
\author{Agostino Cesarano}
\date{January 2024}

\begin{document}

\maketitle
\newtcolorbox{calloutbox}{
    colback=red!5!white,
    colframe=red!75!black,
    fonttitle=\bfseries,
    title=Attenzione
}

\setcounter{part}{3}
\part{Integrali}
\section{Teorema della media integrale}
In una funzione $f(x)$ continua nell'intervallo $[a,b]$ esiste un punto $x_{0}
    \in [a,b]$ tale che è soddisfatta la seguente uguaglianza $$f(x_{0}) =
    \frac{1}{b-a} \int_{a}^{b} f(t) \, dt $$ $$\int_{a}^b f(x) \, dx = f(x_{0})
    \cdot (b-a)$$ \textit{Dato che $f$ è \underline{continua}, vale il Teorema di
    Weierstrass}, e possiamo scrivere che: $$ m \leq f(x) \leq M \text{ } \forall x
    \in [a,b] $$ Per le proprietà degli Integrali (in questo caso la
\textbf{monotonia dell'integrale}) ci permettono di \textbf{integrare tutti i
    membri di questa catena di disuguaglianze mantenendo inalterati i versi delle
    disuguaglianze stesse}: $$ \int_{a}^{b} m \, dt \leq \int_{a}^{b} f(t) \, dt
    \leq \int_{a}^{b} M \, dt $$ Il \textit{primo e terzo integrale sono banali} $$
    \int_{a}^{b} m \, dt = |mx|_{a}^b = mb - ma = m(b-a) \\ \int_{a}^{b} M \, dt =
    |mx|_{a}^b = Mb - Ma = M(b-a) $$ e possiamo riscrivere la catena di
disuguaglianze in questo modo $$ m(b-a) \leq \int_{a}^{b} f(t) \, dt \leq
    M(b-a) $$ e dividendo per $b-a$ che sarà sempre diverso da zero perché ($a \neq
    b$) si ha $$ m \leq \frac{1}{b-a} \int_{a}^b f(t) \, dt \leq M $$ e siamo
arrivati alla conclusione che il valore $\frac{1}{b-a} \int_{a}^b f(t) \, dt$ è
compreso tra $m$ e $M$, e dato che è continua assume in $[a,b]$ tutti i valori
di $m$ e $M$, secondo il (Secondo)Teorema dell'esistenza dei valori intermedi,
quindi esiste sicuramente un $c \in [a,b]$ tale per cui $$ f(c) = \frac{1}{b-a}
    \int_{a}^b f(t) \, dt $$
\section{Teorema fondamentale del calcolo integrale}
Il \textbf{teorema fondamentale del calcolo integrale} stabilisce una relazione
fondamentale tra la derivata e l'integrale di una funzione.

Esso afferma che se una funzione $f(x)$ è una primitiva per un'altra funzione
$F(x)$, allora la derivata dell'integrale definito $\int_a^x f(t) \, dt$ è
uguale alla funzione originale $f(x)$. Formalmente, questo teorema può essere
enunciato così:

Se $F(x)$ è una qualsiasi primitiva di $f(x)$, allora per ogni $x$
nell'intervallo in cui $f(x)$ è continua, abbiamo: $$ F'(x) = f(x) $$ $$ \left(
    \int_a^x f(t) \, dt \right)' = f(x) $$

In altre parole, il teorema fondamentale del calcolo fornisce un metodo per
\textbf{calcolare l'integrale definito di una funzione, fornendo la relazione
    diretta tra la primitiva di una funzione e l'integrale definito di quella
    funzione.}

Si consideri un punto $x_0 \in (a,b)$ e una quantità $h > 0$ /textbf{sufficientemente
piccola}, in modo che $x_0 + h$ sia ancora in $(a,b)$ Grazie alle proprietà
dell'integrale, si ha che $$\int_a^{x_0} f(t)dt + \int_{x_0}^{x_0+h} f(t)dt =
    \int_a^{x_0+h}f(t)dt$$ Inoltre, per definizione di $F(x)$, valgono le seguenti
uguaglianze:$$F(x_0) = \int_a^{x_0} f(t) \, dt \text{ e } F(x_0 + h) =
    \int_a^{x_0+h}f(t) \, dt$$ A questo punto si ottiene facilmente questa
relazione: $$F(x_0+h) - F(x_0) = \int_{x_0}^{x_0 + h}f(t)dt$$ Dato che $f$ è
continua su $(a,b)$, si può applicare il \textit{\underline{Teorema della media
        integrale}}, che assicura che esiste un $c \in (x_0, x_0 + h)$ tale per cui
$$\int_{x_0}^{x_0 + h}f(t) \, dt = f(c) \cdot (x_0 +h - x_0) = f(c) \cdot
    h$$Quindi la relazione può essere rielaborata così: $$F(x_0+h) - F(x_0) = f(c)
    \, h \quad \Rightarrow \quad \frac{F(x_0+h) - F(x_0)}{h} = f(c)$$ Si può far
passare al limite entrambi i membri, ottenendo$$\lim_{h \to 0} \frac{F(x_0+h) -
        F(x_0)}{h} = \lim_{h \to 0}f(c)$$\\ Si ha che $F'(x_0) = \lim_{h \to
        0}f(c)$.\\Analizzando il termine $\lim_{h \to 0}f(c)$, si sfrutta la continuità
di $f$ per affermare che $$\lim_{h \to 0}f(c) = f(L)$$ dove $L = \lim_{h \to
        0}c$ Dato che vale sempre $x_0 \leq c \leq x_0 + h$, e che $\lim_{h \to 0}x_0 =
    x_0$ e $\lim_{h \to 0} ( x_0 + h )= x_0$, allora per il \underline{Teorema del
    confronto} si ha che \[L = \lim_{h \to 0}c = x_0\] e quindi che \[\lim_{h \to 0}f(c) = f(L) = f(x_0)\]
In sostanza si è mostrato che $F'(x_0) = f(x_0)$, il che significa che $F$ è
derivabile, e che calcolare la sua derivata in un qualunque punto $x_0 \in
    (a,b)$ è come calcolare $f$ in tale punto, così come si voleva dimostrare.
\section{(Secondo) Teorema fondamentale del calcolo integrale}
Il \textbf{secondo teorema fondamentale del calcolo integrale} (detta anche
formula fondamentale del calcolo integrale) afferma che se $f$ è una funzione
continua su un intervallo chiuso $[a,b]$ e $G$ è una primitiva di $f$ su
quell'intervallo, allora l'integrale definito di $f$ da $a$ a $b$ può essere
calcolato come la differenza tra i valori di $G$ nei punti $b$ e $a$:

$$
    \int_a^b f(x) \, dx = G(b) - G(a) = G(x)|_{a}^{b}
$$
dove $G$ è una \textbf{Funzione primitiva} di $f(x)$

Questo teorema rappresenta una relazione fondamentale tra il calcolo dell'area
sotto una curva (integrale) e l'antiderivata della funzione che descrive la
curva stessa.

Sia $G$ la primitiva definita da $F(x)= \int_{a}^x f(t) \, dt + C \, \forall c
    \in \mathbb{R}$ allora $$ F(b)- F(a)= \int_{a}^b f(t) \, dt + C - \int_{a}^a
    f(t) \, dt - C $$

visto che le due $C$ si annullano e $\int_{a}^a f(t) \, dt = 0$ rimane che $$
    F(b)- F(a)= \int_{a}^b f(t) \, dt $$ e cosi lo dimostriamo.
\end{document}
